% --------------------------------------------------------------
% This is all preamble stuff that you don't have to worry about.
% Head down to where it says "Start here"
% --------------------------------------------------------------
\documentclass[12pt]{article}
\usepackage[top=2cm,bottom=2cm,left=2.5cm,right=3cm,headsep=8pt,a4paper]{geometry}
\usepackage{listings}
\usepackage{apacite}
\usepackage[usenames]{color}
\definecolor{gray97}{gray}{.97}
\definecolor{gray75}{gray}{.75}
\definecolor{gray45}{gray}{.45}
\definecolor{azul1}{RGB}{141,198,163}
\definecolor{azul2}{RGB}{24,107,122}
\definecolor{verde1}{RGB}{44,186,34}
\usepackage{textcomp}
\lstset{
        frame=Ltb,
        framerule=1pt,
        framextopmargin=3pt,
        framexbottommargin=3pt,
        framexleftmargin=0.4cm,
        framesep=0pt,
        rulesep=.4pt,
		backgroundcolor=\color{gray97},
		rulesepcolor=,
        tabsize=4,
        rulecolor=\color[RGB]{106, 182, 217}, %AZUL
        upquote=true,
        aboveskip={1.5\baselineskip}, %despues de la linea de texto
        columns=fixed,
        showstringspaces=false,
        extendedchars=true,
        breaklines=true,
        prebreak = \raisebox{0ex}[0ex][0ex]{\ensuremath{\hookleftarrow}},
        showtabs=false,
        showspaces=false,
        showstringspaces=false,
        basicstyle=\scriptsize\ttfamily\color[RGB]{39, 100, 46}, %Numeros de lineas, simbolos, puntos y coma y demas
        identifierstyle=\ttfamily\color[RGB]{56, 140, 189}, %variables
        commentstyle=\color[RGB]{62, 179, 101}, %comentarios
        stringstyle=\color[RGB]{247, 165, 42}, %impresiones
        keywordstyle=\bfseries\color[RGB]{237, 118, 150}, %funciones
        %
		numbers=left,
		numbersep=5pt,
		numberstyle=\tiny,
		numberfirstline = false,
		breaklines=true,
		}
\usepackage{amsmath,amsthm,amssymb}
\usepackage[spanish]{babel} %Castellanización
\usepackage[T1]{fontenc} %escribe lo del teclado
\usepackage[utf8]{inputenc} %Reconoce algunos símbolos
\usepackage{lmodern} %optimiza algunas fuentes
\usepackage{graphicx}
\graphicspath{ {images/} }
\usepackage{hyperref} % Uso de links
\newcommand{\N}{\mathbb{N}}
\newcommand{\Z}{\mathbb{Z}}

\newenvironment{theorem}[2][Theorem]{\begin{trivlist}
\item[\hskip \labelsep {\bfseries #1}\hskip \labelsep {\bfseries #2.}]}{\end{trivlist}}
\newenvironment{lemma}[2][Lemma]{\begin{trivlist}
\item[\hskip \labelsep {\bfseries #1}\hskip \labelsep {\bfseries #2.}]}{\end{trivlist}}
\newenvironment{exercise}[2][Exercise]{\begin{trivlist}
\item[\hskip \labelsep {\bfseries #1}\hskip \labelsep {\bfseries #2.}]}{\end{trivlist}}
\newenvironment{problem}[2][Problem]{\begin{trivlist}
\item[\hskip \labelsep {\bfseries #1}\hskip \labelsep {\bfseries #2.}]}{\end{trivlist}}
\newenvironment{question}[2][Question]{\begin{trivlist}
\item[\hskip \labelsep {\bfseries #1}\hskip \labelsep {\bfseries #2.}]}{\end{trivlist}}
\newenvironment{corollary}[2][Corollary]{\begin{trivlist}
\item[\hskip \labelsep {\bfseries #1}\hskip \labelsep {\bfseries #2.}]}{\end{trivlist}}

\newenvironment{solution}{\begin{proof}[Solution]}{\end{proof}}
\begin{document}
% --------------------------------------------------------------
%                         Start here
% --------------------------------------------------------------
\title{Tarea 10}
\author{Hanan Ronaldo Quispe Condori\\ %replace with your name
Procesamiento de Señales II}

\maketitle
\section{Diseñar un filtro IIR utilizando fdatool, filtro pasabajas Chevyshev 1 (rizado banda de paso), $fc=1200Hz,fr=2300Hz,Ap=2dB,Ar=50dB$.}

En la herramienta de $fdatool$, se ingresaron las caracteristicas deseadas del filtro como se indica en la siguiente figura.
\begin{figure}[h]
    \centering
    \includegraphics[width=15cm, height=8cm]{imagenes/fdatool.jpg}
    \caption{Interfaz Gráfica fdatool.}
    \label{fig:fda}
\end{figure}

Al exportar los resultados se obtuvo lo siguiente.
\lstinputlisting[language=matlab]{IIR_filter.fcf}

Del resultado podemos extraer la matriz SOS en la variable $"sos"$ para su conversión a función de transferencia, esto se realizara mediante la función $sos2tf$  como se muestra a continuación.

\lstinputlisting[language=matlab]{uso_ss2tf.m}

Como resultado se tienen los coeficientes de la función de transferencia del filtro.

\lstinputlisting[language=matlab]{result_sos2tf}

Una vez obtenidos estos valores se puede usar la función $freqz$ especificando la frecuencia de muestreo como entrada de la función para obtener la respuesta en frecuencia del filtro diseñado.

\lstinputlisting[language=matlab]{uso_freqz.m}

Esta función da como resultado la siguiente gráfica de respuesta en frecuencia.

%\vspace{8cm}

\begin{figure}[h]
    \centering
    \includegraphics[width=15cm, height=7.7cm]{imagenes/result.jpg}
    \caption{Respuesta en Frecuencia del Filtro.}
    \label{fig:freqz}
\end{figure}

En la figura \ref{fig:freqz} se puede observar que en la frecuencia de corte dada en las especificaciones comienza la transición caracteristica de este tipo de filtro.
\\
Ademas de ello si hacemos zoom en la gráfica de magnitud, se puede observar que el rizado de la banda de paso termina en un valor cercano a la frecuencia de corte dada.

\begin{figure}[h]
    \centering
    \includegraphics[width=15cm, height=4cm]{imagenes/rizado.jpg}
    \caption{Banda de Paso del Filtro.}
    \label{fig:feqz}
\end{figure}


\end{document}